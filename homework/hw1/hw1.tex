\documentclass[UTF8]{ctexart}

\usepackage{amsmath, geometry, amssymb, listings, framed}
\geometry{left=2cm, right=2cm, top=2cm, bottom=2cm}

\title{\vspace{-2cm}数值代数第一次作业}
\author{数学与应用数学(强基计划)2101\quad 王笑同\quad 3210105450}
\date{\today}

\linespread{1.65}

\begin{document}

\maketitle

\pagestyle{plain}

1. 设 $L=\begin{pmatrix}
    l_{11}\\
    \vdots&\ddots\\
    l_{n1}&\cdots&l_{nn}
\end{pmatrix}$ 为下三角矩阵,$\displaystyle\prod_{i=1}^n l_{ii}\neq 0$. 则其逆矩阵也是下三角矩阵,不妨设为 $A=\begin{pmatrix}
    a_{1}&a_2&\cdots&a_n
\end{pmatrix}$. 则由 $LA=I$ 得 $La_k=e_k$,其中 $k=1,2,\dots,n$. 从而可通过前代法解出每个 $a_k$,进而确定 $A$. 算法的伪代码如下:

\begin{framed}
    \begin{lstlisting}[language=C]
        for j = 1 : n
            for k = 1 : n - 1
                A (k, j) = A (k, j) / L (k, k)
                A (k + 1 : n, j) = A (k + 1 : n, j) - 
                                   A (k, j) * L (j + 1 : n, k)
            end
            A (n, j) = A (n, j) / L (n, n)
        end
    \end{lstlisting}
\end{framed}

4. 注意到 $L$ 等价于将 $(2,3,4)^{\mathrm{T}}$ 的第一行的两倍分别加到第二、第三行上,因此
\[L=\begin{pmatrix}
    1\\
    2&1\\
    2&0&1
\end{pmatrix}.\]

7. 设 $L_1=\begin{pmatrix}
    1\\
    -\dfrac{a_{21}}{a_{11}}&1\\
    -\dfrac{a_{31}}{a_{11}}&0&1\\
    \vdots&\vdots&&\ddots\\
    -\dfrac{a_{n1}}{a_{11}}&0&\cdots&0&1
\end{pmatrix}=\begin{pmatrix}
    1&0\\
    -\dfrac{a_1}{a_{11}}&I_{n-1}
\end{pmatrix}$,则 $L_1A=\begin{pmatrix}
    a_{11}&a_1^{\mathrm{T}}\\
    0&A_2
\end{pmatrix}$. 将 $A$ 写作分块矩阵 $A=\begin{pmatrix}
    a_{11}&a_1^\mathrm{T}\\
    a_{1}&A'
\end{pmatrix}$,则有 $A_2=-\dfrac{a_1}{a_{11}}\cdot a_1^{\mathrm{T}}+A'I_{n-1}$. 因为 $A'$ 是对称矩阵,$a_1\cdot a_1^{\mathrm{T}}$ 显然也是 $n\times n$ 的对称矩阵,所以 $A_2$ 是对称矩阵.

8. 沿用前一题的记号,记 $L_1A=(b_{ij})_{n\times n}$,则在矩阵 $A_2$ 中,有
\[b_{ij}=a_{ij}-\dfrac{a_{i1}a_{1i}}{a_{11}},\ i,j=2,3,\dots,n.\]

从而
\[\begin{aligned}
    |b_{ii}|&=\left|a_{ii}-\dfrac{a_{i1}a_{1i}}{a_{11}}\right|>|a_{ii}|-\dfrac{|a_{i1}||a_{1i}|}{|a_{11}|},\\
    \sum_{k=2,k\neq i}^n|b_{ik}|&=\sum_{k=2,k\neq i}^n \left|a_{ik}-\dfrac{a_{i1}a_{1k}}{a_{11}}\right|\leqslant \sum_{k=2,k\neq i}^n |a_{ik}|+\dfrac{|a_{i1}|}{|a_{11}|}\sum_{k=2,k\neq i}^n |a_{1k}|,
\end{aligned}\]

且
\[\sum_{k=2,k\neq i}^n|a_{ik}|<|a_{ii}|-|a_{i1}|,\quad \sum_{k=2,k\neq i}^n |a_{1k}|<|a_{11}|-|a_{i1}|.\]

所以
\[|b_{ii}|>\sum_{k=2,k\neq i}^n |b_{ik}|.\]

10. 沿用第 7 题中的记号,有
\[L_1AL_1^{\mathrm{T}}=\begin{pmatrix}
    a_{11}\\
    &A_2
\end{pmatrix}.\]

因为 $L_1$ 非奇异,所以对任何非零的 $x\in\mathbb{R}^{n-1}$,可构造 $\tilde{x}=(0,x^{\mathrm{T}})^{\mathrm{T}}$ 以及 $y=L_1^{\mathrm{T}}\tilde{x}$,则由 $A$ 是正定矩阵得
\[0<y^{\mathrm{T}}Ay=\tilde{x}^{\mathrm{T}}L_1AL_1^{\mathrm{T}}\tilde{x}=(0,x^{\mathrm{T}})\begin{pmatrix}
    a_{11}\\
    &A_2
\end{pmatrix}\begin{pmatrix}
    0\\x
\end{pmatrix}=x^{\mathrm{T}}A_2 x.\]

由 $x$ 的任意性知 $A_2$ 是正定矩阵.

\end{document}