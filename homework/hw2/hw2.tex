\documentclass[UTF8]{ctexart}

\usepackage{amsmath, geometry, amssymb, listings, framed}
\geometry{left=2cm, right=2cm, top=2cm, bottom=2cm}

\title{\vspace{-2cm}数值代数第二次作业}
\author{数学与应用数学(强基计划)2101\quad 王笑同\quad 3210105450}
\date{\today}

%\linespread{1.5}

\begin{document}

\maketitle

\pagestyle{plain}

1. (1)以 $n$ 阶可逆上三角矩阵 $A$ 为例,只需说明其伴随 $A^*$ 是上三角矩阵,即要说明对任何 $1\leqslant i<j\leqslant n$,有 $M_{ij}=0$,其中 $M_{ij}$ 代表 $a_{ij}$ 的余子式. 注意到 $a_{ij}$ 是上三角元,因此根据代数余子式的定义,$M_{ij}$ 中必然有一个对角元是 $0$,即 $M_{ij}=0$. 由 $i,j$ 的任意性知 $(A^{*})^{\mathrm{T}}$ 是下三角矩阵,从而 $A^*$ 是上三角矩阵.

(2)对矩阵 $A$ 的阶数 $n$ 做归纳. 当 $n=1$ 时,结论自明. 现设结论对大小为 $n<m$ 的单位上三角阵成立,欲证明结论对 $n=m$ 的单位上三角阵成立. 记 $A^{-1}=(b_{ij})_{n\times n}$. 此时,根据归纳假设,$A$ 的 $(m-1)$ 阶顺序主子阵的逆是单位上三角阵,则 $\det A^{-1}=b_{mm}$. 由 $\det A^{-1}\cdot \det A=1$ 即知 $b_{mm}=1$,即 $A^{-1}$ 是单位上三角矩阵.

\quad

2. 类似熟知的 Gauss 变换,对 $n$ 阶矩阵 $A$,在第 $k\,(1\leqslant k\leqslant n-1)$ 步消元时,令
\[U_k=\begin{pmatrix}
    1\\
    &1\\
    &&\ddots&-u_{kj}&\\
    &&&1\\
    &&&&1
\end{pmatrix},\]

其中 $u_{kj}=\dfrac{a_{jk}^{(n-k)}}{a_{kk}^{(n-k)}}$,$j=1,2,\dots,k-1$. 则
\[L_k\cdots L_{n-k}A^{(n-k)}=\begin{pmatrix}
    A_{22}^{(n-k)}&O\\
    A_{12}^{(n-k)}&A_{11}&{(n-k)}
\end{pmatrix}.\]

记 $L=U_1U_2\cdots U_nA$,则 $L$ 为下三角矩阵;$U=U_n^{-1}U_{n-1}^{-1}\cdots U_2^{-1}=I+u_ne_n^{\mathrm{T}}+\cdots+u_2e_2^{\mathrm{T}}$ 位单位上三角矩阵. 此时成立 $A=UL$. 特别地,若 $A$ 可逆,则 $a_{ii}^{(i-1)},\ i=1,2,\dots,k$ 均非零,从而对角线上所有顺序主子阵非奇异,由此即得分解的唯一性.

\quad

3. (1)将 $A$ 左乘 $L_1=\begin{pmatrix}
    1\\
    -2 & 1\\
    -3 & 0 & 1
\end{pmatrix}$,得到 $L_1A=\begin{pmatrix}
    1 & 4 & 7\\
    0 & -3 & -6\\
    0 & -6 & -11\\
\end{pmatrix}$;将 $L_1A$ 左乘 $L_2=\begin{pmatrix}
    1\\
    0&1\\
    0&-2&1
\end{pmatrix}$,得到 $L_2(L_1A)=\begin{pmatrix}
    1 & 4 & 7\\
    0 & -3 & -6\\
    0 & 0 & 1
\end{pmatrix}$. 从而所求 LU 分解为
\[L=(L_2L_1)^{-1}=\begin{pmatrix}
    1 & 0 & 0\\
    2 & 1 & 0\\
    3 & 2 & 1
\end{pmatrix},\quad U=\begin{pmatrix}
    1 & 4 & 7\\
    0 & -3 & -6\\
    0 & 0 & 1
\end{pmatrix}.\]

利用前代法,解方程组 $Ly=b$ 得 $y=(1,-1,0)^{\mathrm{T}}$,再利用回代法解方程组 $Ux=y$ 得 $x=\left(-\dfrac13,\dfrac13,0\right)$.

(2)第 1 步消元,选定第 1 列的第 3 个元素为主元,交换第 1 行和第 3 行得 $\begin{pmatrix}
    3 & 6 & 10\\
    2 & 5 & 8\\
    1 & 4 & 7
\end{pmatrix}$,此时 $L_1=\begin{pmatrix}
    1 \\
    -\dfrac{2}{3} & 1 \\
    -\dfrac{1}{3} & 0 & 1
\end{pmatrix}$,消元后 $L_1A=\begin{pmatrix}
    3 & 6 & 10\\
    0 & 1 & \dfrac43\\
    0 & 2 & \dfrac{11}{3}
\end{pmatrix}$. 第 2 步消元,选定第 2 列的第 2 个元素为主元,交换 $L_1A$ 的第 2 行和第 3 行得 $\begin{pmatrix}
    3 & 6 & 10 \\
    0 & 2 & \dfrac{11}{3} \\
    0 & 1 & \dfrac43
\end{pmatrix}$. 此时 $L_2=\begin{pmatrix}
    1 \\
    0 & 1 \\
    0 & -\dfrac12 & 1
\end{pmatrix}$,消元后 $L_2(L_1A)=\begin{pmatrix}
    3 & 6 & 10 \\
    0 & 2 & \dfrac{11}{3} \\
    0 & 0 & -\dfrac12
\end{pmatrix}$. 从而所求列主元分解为
\[P=\begin{pmatrix}
    0 & 0 & 1 \\
    1 & 0 & 0 \\
    0 & 1 & 0 \\
\end{pmatrix},\quad L=\begin{pmatrix}
    1 \\
    \dfrac13 & 1 \\
    \dfrac23 & \dfrac12 & 1
\end{pmatrix},\quad U=L_2(L_1A)=\begin{pmatrix}
    3 & 6 & 10 \\
    0 & 2 & \dfrac{11}{3} \\
    0 & 0 & -\dfrac12
\end{pmatrix}.\]

用前代法解方程组 $Ly=Pb$ 得 $y=\left(1,\dfrac23,0\right)^{\mathrm{T}}$,用回代法解方程组 $Ux=y$ 得 $x=\left(-\dfrac13,\dfrac13,0\right)$.

(3)第 1 步消元,选定第 3 行第 3 列的元素为主元,交换第 3 行和第 1 行、第 3 列和第 1 列得 $\begin{pmatrix}
    10 & 6 & 3 \\
    8 & 5 & 2 \\
    7 & 4 & 1
\end{pmatrix}$. 此时 $L_1=\begin{pmatrix}
    1 \\
    -\dfrac45 & 1 \\
    -\dfrac7{10} & 0 & 1
\end{pmatrix}$,消元得 $L_1A=\begin{pmatrix}
    10 & 6 & 3\\
    0 & \dfrac15 & -\dfrac25 \\
    0 & -\dfrac15 & -\dfrac{11}{10}
\end{pmatrix}$. 第 2 步消元,主元为第 3 行第 3 列的元素,交换第 3 行和第 2 行、第 3 列和第 2 列得 $\begin{pmatrix}
    10 & 3 & 6 \\
    0 & -\dfrac{11}{10} & -\dfrac15 \\
    0 & -\dfrac25 & \dfrac15
\end{pmatrix}$. 此时 $L_2=\begin{pmatrix}
    1 \\
    0 & 1 \\
    0 & -\dfrac{4}{11} & 1
\end{pmatrix}$,从而 $L_2(L_1A)=\begin{pmatrix}
    10 & 3 & 6\\
    0 & -\dfrac{11}{10} & -\dfrac15 \\
    0 & 0 & \dfrac{7}{55}
\end{pmatrix}$. 于是
\[P=\begin{pmatrix}
    0 & 0 & 1 \\
    1 & 0 & 0 \\
    0 & 1 & 0 \\
\end{pmatrix},\quad Q=\begin{pmatrix}
    0 & 0 & 1 \\
    1 & 0 & 0 \\
    0 & 1 & 0 \\
\end{pmatrix},\quad L=\begin{pmatrix}
    1 & 0 & 0 \\
    \dfrac7{10} & 1 & 0 \\
    \dfrac{8}{10} & \dfrac{4}{11} & 1
\end{pmatrix},\quad U=\begin{pmatrix}
    10 & 3 & 6\\
    0 & -\dfrac{11}{10} & -\dfrac15 \\
    0 & 0 & \dfrac{7}{55}
\end{pmatrix}.\]

用前代法求解 $Ly=Pb$ 得 $y$,用回代法解 $Uz=y$ 得 $z$,最后由 $x=Qz$ 得 $x=\left(-\dfrac13,\dfrac13,0\right)$.

\quad

\quad

\textbf{上机习题见 program 文件夹,程序是在 Linux 下编译的:)}

\end{document}