\documentclass[UTF8]{ctexart}

\usepackage{amsmath, geometry, amssymb, listings, framed}
\geometry{left=2cm, right=2cm, top=2cm, bottom=2cm}

\title{\vspace{-2cm}数值代数第三次作业}
\author{数学与应用数学(强基计划)2101\quad 王笑同\quad 3210105450}
\date{\today}

\linespread{1.65}

\begin{document}

\maketitle

\pagestyle{plain}

1. 利用平方分解法,可得 $LL^{\mathrm{T}}=A$,其中
\[L=\begin{pmatrix}
    2\\
    -1 & 3\\
    2 & 0 & 2\\
    1 & -2 & 1 & 1
\end{pmatrix}.\]

用前代法解方程组 $Ly=b$,得 $y=(4,2,4,2)^{\mathrm{T}}$;再利用回代法解 $L^{\mathrm{T}}x=y$,得 $x=(1,2,1,2)^{\mathrm{T}}$.

\quad

2. 利用改进的平方分解法,可得 $LDL^{\mathrm{T}}=A$,其中
\[L=\begin{pmatrix}
    1\\
    2&1\\
    3&4&1
\end{pmatrix},\quad D=\begin{pmatrix}
    10\\
    &5\\
    &&1
\end{pmatrix}.\]

用前代法解方程 $Ly=b$ 得 $y=(10,-15,-1)^{\mathrm{T}}$,简单计算得 $Dz=y$ 的解是 $z=(1,-3,1)^{\mathrm{T}}$,用回代法求方程 $L^{\mathrm{T}}x=z$ 得 $x=(2,1,-1)^{\mathrm{T}}$.

\quad

3. 设 $A$ 有分解
\[A=\begin{pmatrix}
    \alpha_1\\
    \gamma_2&\alpha_2\\
    &\gamma_3&\alpha_3
\end{pmatrix}\begin{pmatrix}
    1&\beta_1\\
    &1&\beta_2\\
    &&1
\end{pmatrix},\]

则由追赶法公式
\[\begin{cases}
    \alpha_i=a_i-c_i,\\
    \beta_i=\dfrac{b_i}{a_i-c_i},\\
    \gamma_i=c_i
\end{cases}\]

得到
\[A=\begin{pmatrix}
    -4\\
    2&-4\\
    &2&8
\end{pmatrix}\begin{pmatrix}
    1&-1\\
    &1&-1\\
    &&1
\end{pmatrix}.\]

依次用前代法和回代法解得 $x=(1,-1,0)^{\mathrm{T}}$.

\newpage

4. 写出增广矩阵,依次进行行列变换得到(过程太长就不写了QAQ)

\[A^{-1}=\begin{pmatrix}
    -\frac{4}{85} & \frac{10}{17} & -\frac{23}{85} & -\frac{16}{17} \\
    \frac{33}{85} & -\frac{6}{17} & \frac{41}{85} & \frac{13}{17} \\
    \frac{19}{85} & \frac{5}{17} & -\frac{3}{85} & -\frac{8}{17} \\
    -\frac{3}{85} & -\frac{1}{17} & \frac{4}{85} & \frac{5}{17}
\end{pmatrix}\]

\textbf{上机习题见 program 文件夹,其中的 result.md 给出了程序的计算结果和我对作业的解答. 程序是在 Linux 下编译的:)}

\end{document}