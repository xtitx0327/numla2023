\documentclass[UTF8]{ctexart}

\usepackage{amsmath, geometry, amssymb, listings, framed}
\geometry{left=2cm, right=2cm, top=2cm, bottom=2cm}

\title{\vspace{-2cm}数值代数第二次作业}
\author{数学与应用数学(强基计划)2101\quad 王笑同\quad 3210105450}
\date{\today}

\linespread{1.65}

\begin{document}

\maketitle

\pagestyle{plain}

1. (1)以 $n$ 阶可逆上三角矩阵 $A$ 为例,只需说明其伴随 $A^*$ 是上三角矩阵,即要说明对任何 $1\leqslant i<j\leqslant n$,有 $M_{ij}=0$,其中 $M_{ij}$ 代表 $a_{ij}$ 的余子式. 注意到 $a_{ij}$ 是上三角元,因此根据代数余子式的定义,$M_{ij}$ 中必然有一个对角元是 $0$,即 $M_{ij}=0$. 由 $i,j$ 的任意性知 $(A^{*})^{\mathrm{T}}$ 是下三角矩阵,从而 $A^*$ 是上三角矩阵.

(2)对矩阵 $A$ 的阶数 $n$ 做归纳. 当 $n=1$ 时,结论自明. 现设结论对大小为 $n<m$ 的单位上三角阵成立,欲证明结论对 $n=m$ 的单位上三角阵成立. 记 $A^{-1}=(b_{ij})_{n\times n}$. 此时,根据归纳假设,$A$ 的 $(m-1)$ 阶顺序主子阵的逆是单位上三角阵,则 $\det A^{-1}=b_{mm}$. 由 $\det A^{-1}\cdot \det A=1$ 即知 $b_{mm}=1$,即 $A^{-1}$ 是单位上三角矩阵.

2. 

\end{document}